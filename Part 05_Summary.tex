\documentclass[10pt,landscape]{article}
\usepackage{multicol}
\usepackage{calc}
\usepackage{ifthen}
\usepackage[landscape]{geometry}
\usepackage{amsmath,amsthm,amsfonts,amssymb}
\usepackage{color,graphicx,overpic}
\usepackage{hyperref}


\pdfinfo{
  /Title (Part_05_Summary.pdf)
  /Creator (TeX)
  /Producer (pdfTeX 1.40.0)
  /Author (Khoo Wu Zhe Samuel)
  /Subject (Survey Methodology)
  /Keywords (pdflatex, latex,pdftex,tex)}

% This sets page margins to .5 inch if using letter paper, and to 1cm
% if using A4 paper. (This probably isn't strictly necessary.)
% If using another size paper, use default 1cm margins.
\ifthenelse{\lengthtest { \paperwidth = 11in}}
    { \geometry{top=.5in,left=.5in,right=.5in,bottom=.5in} }
    {\ifthenelse{ \lengthtest{ \paperwidth = 297mm}}
        {\geometry{top=1cm,left=1cm,right=1cm,bottom=1cm} }
        {\geometry{top=1cm,left=1cm,right=1cm,bottom=1cm} }
    }

% Turn off header and footer
\pagestyle{empty}

% Redefine section commands to use less space
\makeatletter
\renewcommand{\section}{\@startsection{section}{1}{0mm}%
                                {-1ex plus -.5ex minus -.2ex}%
                                {0.5ex plus .2ex}%x
                                {\normalfont\large\bfseries}}
\renewcommand{\subsection}{\@startsection{subsection}{2}{0mm}%
                                {-1explus -.5ex minus -.2ex}%
                                {0.5ex plus .2ex}%
                                {\normalfont\normalsize\bfseries}}
\renewcommand{\subsubsection}{\@startsection{subsubsection}{3}{0mm}%
                                {-1ex plus -.5ex minus -.2ex}%
                                {1ex plus .2ex}%
                                {\normalfont\small\bfseries}}
\makeatother

% Define BibTeX command
\def\BibTeX{{\rm B\kern-.05em{\sc i\kern-.025em b}\kern-.08em
    T\kern-.1667em\lower.7ex\hbox{E}\kern-.125emX}}

% Don't print section numbers
\setcounter{secnumdepth}{0}


\setlength{\parindent}{0pt}
\setlength{\parskip}{0pt plus 0.5ex}

%My Environments
\newtheorem{example}[section]{Example}
% -----------------------------------------------------------------------

\begin{document}
\raggedright
\footnotesize
\begin{multicols}{3}


% multicol parameters
% These lengths are set only within the two main columns
%\setlength{\columnseprule}{0.25pt}
\setlength{\premulticols}{1pt}
\setlength{\postmulticols}{1pt}
\setlength{\multicolsep}{1pt}
\setlength{\columnsep}{2pt}

\begin{center}
     \Large{\underline{Part 5 - Stratified Simple Random}} \\
     \Large{\underline{Sampling}}
\end{center}

\section{\underline{5.1 Description}}

\textbf{Stratified sampling:}

Sampling independently from each stratum where the population is partitioned into $H$ subpopulations (strata), assuming that it is possible to sample directly from each stratum.

\textbf{Stratified SRS:}

A special case of stratified sampling that uses simple random sampling for selecting units from each stratum.

\textbf{Reasons for using stratified sampling:}
\begin{enumerate}
  \item We are interested in certain subpopulations of the whole population
  \item Administrative convenience
  \item Different approaches to sampling may be required for different parts of the population which should be treated as different strata
  \item Ensure that every subpopulation is represented in the final sample
  \item Stratification may produce more efficient estimates of the characteristics of the whole population. This happens when a heterogeneous population is divided into strata that are internally homogeneous.
\end{enumerate}

\section{\underline{5.2 Notation}}
For $h = 1,...,H$, where $H$ is the number of strata:
\begin{itemize}
  \item[$N_{h}$] size of stratum $h$
  \item[$t_{h}$] = $\sum_{j=1}^{N_{h}}y_{hj}$
  \item the total of variable $y$ in stratum $h$, where $y_{hj}$ is the y-value of unit $j$ in stratum $h$
  \item[$\bar{y}_{U_{h}}$] = $\frac{t_{h}}{N_{h}}$
  \item the mean of variable $y$ in stratum $h$
  \item[$S^{2}_{h}$] = $\frac{1}{N_{h} - 1}\sum_{j=1}^{N_{h}}(y_{hj} - \bar{y}_{U_{h}})^{2}$
  \item the variance of variable $y$ in stratum $h$
\end{itemize}

Sample analogues:
\begin{itemize}
  \item[$\textbf{S}_{h}$] sample drawn from stratum $h$
  \item[$n_{h}$] size of $\textbf{S}_{h}$ (unlike post-stratification, $n_{h}$ is fixed)
  \item[$\bar{y}_{h}$] = $\frac{1}{n_{h}}\sum_{j \in \textbf{S}_{h}}y_{hj}$
  \item[$s_{h}^{2}$] = $\frac{1}{n_{h} - 1}\sum_{j \in \textbf{S}_{h}}(y_{hj} - \bar{y}_{h})^{2}$
\end{itemize}

Relating overall population quantities to stratum quantities:

\begin{itemize}
  \item Overall population size: $N=N_{1}+...+N_{H}$
  \item Total sample size: $n=n_{1}+...+n_{H}$
  \item Overall population total: $t=t_{1}+...+t_{H}$
  \item Overall population mean: $\bar{y}_{U} = \sum_{h=1}^{H}W_{h}\bar{y}_{U_{h}}$
\end{itemize}

where $W_{h} = \frac{N_{h}}{N}$ is the stratum weight

\section{\underline{5.3 Estimation of population mean,}}
\section{\underline{total and proportion}}

\textbf{Stratified mean:}

\begin{center}
  $\bar{y}_{str} = \sum_{h=1}^{H}W_{h}\bar{y}_{h}$
\end{center}

\begin{itemize}
  \item An unbiased estimator of $\bar{y}_{U}$
\end{itemize}

\textbf{Variance of $\bar{y}_{str}$:}

\begin{equation}
  \begin{split}
    V(\bar{y}_{str}) &= \sum_{h=1}^{H}W_{h}^{2}V(\bar{y}_{h}) $\hspace{5} by independence$\\
    &= \sum_{h=1}^{H}W_{h}^{2}(1-\frac{n_{h}}{N_{h}})\frac{1}{n_{h}}S_{h}^{2} \\
    &= \sum_{h=1}^{H}W_{h}^{2}(\frac{1}{n_{h}} - \frac{1}{N_{h}})S_{h}^{2}
  \end{split}
\end{equation}

\textbf{Estimated variance of $\bar{y}_{str}$:}

\begin{equation}
  \hat{V}(\bar{y}_{str}) = \sum_{h=1}^{H}W_{h}^{2}(1-\frac{n_{h}}{N_{h}})\frac{1}{n_{h}}s_{h}^{2}
\end{equation}

\begin{itemize}
  \item $SE(\bar{y}_{str}) = \sqrt{\hat{V}(\bar{y}_{str})}$
\end{itemize}

\textbf{Estimation of proportion:} a special case of estimating the means when $y$ is 0-1.

\vspace{5}

\textbf{Estimation of population total:}

Two methods to obtain the same estimator.

\vspace{5}

Number-raised stratified mean:
\begin{center}
$\hat{t}_{str} = N\bar{y}_{str}$
\end{center}

Number-raised within stratum:
\begin{center}
$\hat{t}_{str} = N_{1}\bar{y}_{1} + ... + N_{H}\bar{y}_{H}$
\end{center}

\begin{itemize}
  \item $SE(\hat{t}_{str}) = N\{SE(\bar{y}_{str})\}$
\end{itemize}

\section{\underline{5.4 Design problems}}

\subsection{5.4.1 Sample allocation}

For a given total sample size $n$, how should we choose $n_{1},...,n_{H}$?

\vspace{5}

\textbf{1. Proportional allocation:}

\begin{center}
$n_{h} = \frac{n}{N}N_{h}$
\end{center}

\textbf{Variance of $\bar{y}_{str}$ (proportional):}
\begin{equation}
  \begin{split}
    V_{prop}(\bar{y}_{str}) &= \sum_{h=1}^{H}W_{h}^{2}(1-\frac{n}{N})\frac{1}{nW_{h}}S_{h}^{2} \\
    &= (1-\frac{n}{N})\frac{1}{n}\sum_{h=1}^{H}W_{h}S_{h}^{2}
  \end{split}
\end{equation}

(3) may not be exact as it assumes $n_{h}$ is an integer. This may not be the case in practice. However, it is still useful for interpretation and in theoretical comparison.

\vspace{5}

\textbf{Proportional stratified SRS vs SRS:}

\vspace{5}

Ignoring term of order $\frac{1}{N_{h}-1}$ and $\frac{1}{N-1}$,
\begin{center}
$V_{prop}(\bar{y}_{str}) \leq V_{SRS}(\bar{y})$
\end{center}

* strictly less than unless all the stratum means are equal.

\vspace{5}

Proof:

\vspace{5}

Required to show that $S^{2} \geq \sum_{h=1}^{H}W_{h}S_{h}^{2}$.

\begin{equation}
  \begin{split}
    LHS &= S^{2} \\
    &= \frac{N}{N-1}\frac{1}{N}\sum_{h=1}^{H}\sum_{j=1}^{N_{h}}(y_{hj - \bar{y}_{U}})^{2} \\
    &= (1+\frac{1}{N-1})\frac{1}{N}\sum_{h=1}^{H}\sum_{j=1}^{N_{h}}(y_{hj - \bar{y}_{U}})^{2} \\
    &\approx \frac{1}{N}\sum_{h=1}^{H}\sum_{j=1}^{N_{h}}(y_{hj - \bar{y}_{U}})^{2} $\hspace{5} ignoring term of order$
  \end{split}
\end{equation}

Since
\begin{center}
    $S_{h}^{2} &\approx \frac{1}{N_{h}}\sum_{j=1}^{N_{h}}(y_{U_{h}})^{2}$\hspace{5} ignoring term of order
\end{center}

Hence,
\begin{center}
  $RHS \approx \frac{1}{N}\sum_{h=1}^{H}\sum_{j=1}^{N_{h}}(y_{hj} - \bar{y}_{U_{h}})^{2}$
\end{center}

\begin{itemize}
  \item LHS = Total sum of squares
  \item RHS = Residual sum of squares
\end{itemize}

Since Total SS = Between group SS + Residual SS,
\begin{center}
    LHS $\geq$ RHS
\end{center}

\textbf{2. Neyman allocation:}

Optimal among all allocations with the same total sample size. Choose $n_{h}$ to minimise $V(\bar{y}_{str})$ given by (1) subject to the constraint $n_{1} + ... + n_{H} = n$

Using method of Lagrange Multiplier,

\begin{equation}
  \begin{split}
    L &= \sum_{h=1}^{H}W_{h}^{2}(1-\frac{n_{h}}{N_{h}})\frac{1}{n_{h}}S_{h}^{2} + \lamda(n_{1} + ... + n_{H} - n) \\
    &= \sum_{h=1}^{H}W_{h}^{2}(\frac{1}{n_{h}}-\frac{1}{N_{h}})\frac{1}{n_{h}}S_{h}^{2} + \lambda(n_{1} + ... + n_{H} - n)
  \end{split}
\end{equation}

Setting $\frac{\delta L}{\delta n_{h}} &= -\frac{1}{n_{h}^{2}}W_{h}^{2}S_{h}^{2} + \lambda = 0$,

\begin{center}
  $n_{h} = cN_{h}S_{h}$
\end{center}

We should take a large sample from stratum $h$ if the stratum size $N_{h}$ and/or the stratum deviation $S_{h}$ is large.

\vspace{5}

Since
\begin{center}
  $c = \frac{n}{N_{1}S_{1} + ... + N_{H}S_{H}}$
\end{center}
Hence,
\begin{equation}
  \begin{split}
    n_{h} &= cN_{h}S_{h} \\
    &= \frac{nN_{h}S_{h}}{N_{1}S_{1} + ... + N_{H}S_{H}} \\
    &= \frac{nW_{h}S_{h}}{W_{1}S_{1} + ... + W_{H}S_{H}} \\
    &= \frac{nW_{h}S_{h}}{S_{WS}}
  \end{split}
\end{equation}

Where $S_{WS} = \sum_{h=1}^{H}W_{h}S_{h}$, the weighted sum of stratum standard deviation.

\vspace{5}

\textbf{Variance of $\bar{y}_{str}$ (optimal):}
\begin{equation}
  \begin{split}
    V_{opt}(\bar{y}_{str}) &= \sum_{h=1}^{H}W_{h}^2(\frac{1}{n_{h}} - \frac{1}{N_{h}})S_{h}^{2} \\
    &= \sum_{h=1}^{H}W_{h}^{2}S_{h}^{2}\frac{S_{WS}}{nW_{h}S_{h}} - \sum_{h=1}^{H}\frac{1}{N}W_{h}S_{h}^{2} \\
    &= \frac{(\sum W_{h}S_{h})^{2}}{n} - \frac{\sum W_{h}S_{h}^{2}}{N}
  \end{split}
\end{equation}

\textbf{Neyman allocation vs proportional allocation:}

\begin{equation}
  \begin{split}
    V_{prop}(\bar{y}_{str}) - V_{opt}(\bar{y}_{str}) &= \frac{1}{n}\left\{\sum_{h=1}^{H}W_{h}S_{h}^{2}-\left(\sum_{h=1}^{H}W_{h}S_{h}\right)^{2}\right\} \\
    &= \frac{1}{n}Var(X)
  \end{split}
\end{equation}

of a discrete random variable X that takes on the value $S_{h}$ with probability $W_{h}$. Hence,

\begin{center}
  $V_{opt}(\bar{y}_{str}) \leq V_{prop}(\bar{y}_{str})$
\end{center}

\textbf{Note:} The word "optimal" is deceptive because the optimal allocation for estimating the population mean $\bar{y}_{U}$ of variable $y$ may not be optimal for estimating the mean $\bar{z}_{U}$ of another vriable $z$.

\subsection{5.4.2 Setting stratum boundaries}

When stratifying according to an auxiliary variable, we have to set stratum boundaries. We use the \textbf{square root frequency} method.

\vspace{5}

Let $x$ be an auxiliary variable where the value of $x$ is known for each unit in the population and $x$ is highly correlated with $y$.

\begin{enumerate}
  \item Partition the range of $x$ into a large number of narrow equal-width intervals.
  \item Calculate the population frequency $f$ within each interval.
  \item set stratum boundaries in such a way that the sums of $\sqrt{f}$ within the stratum are roughly equal.
\end{enumerate}

The boundaries obtained using the $\sqrt{f}$ method are optimal for the estimation of $\bar{x}_{U}$.

\vspace{5}

They should also do well for estimating $\bar{y}_{U}$ if variable $x$ is well correlated with $y$.

\subsection{5.4.3 Number of stratas}

It appears at first sight that increasing the number of stratas is a good idea as this is likely to lead to stratas that are internally homogeneous.

\vspace{5}

However, we cannot keep on increasing the number of strats indefinitely as this will increase administrative costs and lead to small sample size (and large variance) within the stratum.

\vspace{5}

Assuming that the relationship between $x$ and $y$ follows a simple linear regression model, if we stratify according to $x$ using the $\sqrt{f}$ method and allocate the sample using optimum allocation, then it can be proven that $V(\bar{y}_{str})$ is approxmately proportional to
\begin{center}
  $V(H) = \frac{\rho^{2}}{H^{2}} + (1-\rho^{2})$
\end{center}

where $H$ is the number of strata and $\rho$ is the correlation between $x$ and $y$.

\vspace{5}

Then choose the number of strata according to the reduction in variance when adding an additional strata.

\section{\underline{5.5 Post-stratifying stratified SRS}}

Refer to Q5 of Tutorial 7.

\section{\underline{5.6 Ratio estimation under stratified}}
\section{\underline{sampling}}

In section 3.1, we consider the use of auxiliary information in estimation and defined the ratio estimator. With stratification, there are 2 ways to construct the ratio estimates: combined or separate.

\subsection{5.6.1 Combined Ratio estimator}

Since
\begin{itemize}
  \item[$\hat{t}_{y, str}$] = $\sum_{j=1}^{N_{h}}\bar{y}_{hj}$ estimates $t_{y}$
  \item[$\hat{t}_{x, str}$] = $\sum_{j=1}^{N_{h}}\bar{x}_{hj}$ estimates $t_{x}$
\end{itemize}

Hence,
\begin{center}
  $\hat{B}_{str} = \frac{\hat{t}_{y, str}}{\hat{t}_{x, str}} = \frac{\bar{y}_{str}}{\bar{x}_{str}}$ estimates B
\end{center}

Since $t_{x}$ is known, we can estimate $t_{y} = Bt_{x}$ by
\begin{center}
  $\hat{t}_{yrc} = \hat{B}_{str}t_{x}$
\end{center}
the combined ratio estimator of the total $t_{y}$.

\vspace{5}

\textbf{Approximate mean and variance:}
\begin{equation}
  \begin{split}
    \hat{B}_{str} - B &= \frac{\bar{y}_{str} - B\bar{x}_{str}}{\bar{x}_{str}} \\
    &= \frac{\bar{y}_{str} - B\bar{x}_{str}}{\bar{x}_{U}}\left(\frac{\bar{x}_{U}}{\bar{x}_{str}} -1 +1 \right) \\
    &= \frac{\bar{y}_{str} - B\bar{x}_{str}}{\bar{x}_{U}} + \frac{\bar{y}_{str} - B\bar{x}_{str}}{\bar{x}_{U}}\left(\frac{\bar{x}_{U}}{\bar{x}_{str}} -1 \right) \\
    &\approx \frac{\bar{y}_{str} - B\bar{x}_{str}}{\bar{x}_{U}} $\hspace{5} since $\frac{\bar{x}_{U}}{\bar{x}_{str}}$ goes to 1$ \\
    &= \frac{\bar{z}_{str}}{\bar{x}_{U}}
  \end{split}
\end{equation}

where $z$ is a new variable,
\begin{center}
  $z_{hi} = y_{hi} - Bx_{hi}$
\end{center}
for unit $i$ $(i=1,...,N_{h})$ in stratum $h$ $(h=1,...,H)$

\vspace{5}

Multiplying $t_{x}$ to (9),
\begin{center}
  $\hat{t}_{yrc} - t_{y} \approx N\bar{z}_{str} = \hat{t}_{z,str}$
\end{center}

If the samples sizes $n_{1}, ... , n_{H}$ are large. Hence,
\begin{center}
  $\hat{t}_{yrc} \approx t_{y} + \hat{t}_{z,str}$
\end{center}

It follows that

\begin{equation}
  \begin{split}
    E(\hat{t}_{yrc}) &\approx t_{y} + E(\hat{t}_{z,str}) \\
    &= t_{y} + t_{z} \\
    &= t_{y} $\hspace{5} since $t_{z} = t_{y} - Bt_{x} = 0$ $
  \end{split}
\end{equation}

\begin{equation}
  \begin{split}
    V(\hat{t}_{yrc}) &\approx V(\hat{t}_{z,str}) \\
    &= \sum_{h=1}^{H}N_{h}^{2}(1-\frac{n_{h}}{N_{h}})\frac{1}{n_{h}}S_{zh}^{2}
  \end{split}
\end{equation}

where,
\begin{center}
  $S_{zh}^{2} = S_{yh}^{2} - 2BS_{xyh} + B^{2}S_{xh}^{2}$
\end{center}
is the stratum variance of the variable $z$ in stratum $h$. Derivation is similar to equation (5) in Part 3.
\vspace{5}

\textbf{Note:} even though $\bar{z}_{U} = \bar{y}_{U} - B\bar{x}_{U} = 0$, $\bar{z}_{U_{h}} = \bar{y}_{U_{h}} - B\bar{x}_{U_{h}} \neq 0$ in general

\vspace{5}
For sample data,
\begin{center}
  $s_{zh}^{2} = s_{yh}^{2} - 2\hat{B}_{str}s_{xyh} + \hat{B}^{2}_{str}s_{xh}^{2}$
\end{center}
Similar to section 3.1,
\begin{equation}
  \hat{V}(\hat{t}_{yrc}) &= \left(\frac{\bar{x}_{U}}{\bar{x}_{str}}\right)^{2}\sum_{h=1}^{H}N_{h}^{2}(1-\frac{n_{h}}{N_{h}})\frac{1}{n_{h}}s_{zh}^{2}
\end{equation}

\subsection{5.6.2 Separate Ratio estimator}

If stratum totals $t_{x1}, ... , t_{xH}$ of the auxiliary variable are known, we are in a position to apply the ratio method separately within each stratum.

\begin{center}
  $\hat{t}_{yhr} = \frac{\bar{y}_{h}}{\bar{x}_{h}} = \hat{B}_{h}t_{xh}$ is an estimate of $t_{yh}$
\end{center}

Hence, the separate ratio estimator of $t_{y}$ is:

\begin{center}
  $\hat{t}_{yrs} = \sum_{h=1}^{H}\hat{t}_{yhr} = \sum_{h=1}^{H}\hat{B}_{h}N_{h}\bar{x}_{U_{h}}$
\end{center}

\textbf{Approximate mean and variance:}

\vspace{5}

Let $\hat{B}_{h} = \frac{\bar{y}_{h}}{\bar{x}_{h}}$, $B_{h} = \frac{\bar{y}_{U_{h}}}{\bar{x}_{U_{h}}}$.

\begin{center}
  $t_{y} = \sum_{h=1}^{H}t_{yh} = \sum_{h=1}^{H}B_{h}N_{h}\bar{x}_{U_{h}}$
\end{center}

Hence,

\begin{equation}
  \begin{split}
  \hat{t}_{yrs}-t_{y} &= \sum_{h=1}^{H}N_{h}\bar{x}_{U_{h}}(\hat{B}_{h} - B_{h}) \\
  &\approx \sum_{h=1}^{H}N_{h}\bar{u}_{h}
\end{split}
\end{equation}

Using the approximation:
\begin{center}
  $\hat{B}_{h} - B_{h} \approx \frac{\bar{y}_{h} - B_{h}\bar{x}_{h}}{\bar{x}_{U_{h}}} = \frac{\bar{u}_{h}}{\bar{x}_{U_{h}}}$
\end{center}

\begin{itemize}
  \item See equation (22) of Part 2 for details.
  \item $\bar{u}_{h}$ is the stratum sample mean of the variable $u$ defined by $u_{hi} = y_{hi} - B_{h}x_{hi}$
\end{itemize}

Hence $E(\hat{t}_{yrs}) \approx t_{y}$ and
\begin{center}
  $V(\hat{t}_{yrs}) \approx \sum_{h=1}^{H}N_{h}^{2}(1-\frac{n_{h}}{N_{h}})\frac{1}{n_{h}}S_{uh}^{2}$
\end{center}

where
\begin{center}
  $S_{uh}^{2} = S_{yh}^{2} - 2B_{h}S_{xyh} + B_{h}^{2}S_{xh}^{2}$
\end{center}
is the stratum variance of the variable $u$.

\begin{equation}
  \begin{split}
    s_{uh}^{2} &= s_{yh}^{2} - 2B_{h}s_{xyh} + B_{h}^{2}s_{xh}^{2} \\
    &= \frac{\sum_{i \in \textbf{S}_{h}}(y_{hi}-\hat{B}_{h}x_{hi})^{2}}{n_{h} - 1}
  \end{split}
\end{equation}

Hence,
\begin{equation}
    \hat{V}(\hat{t}_{yrs}) = \sum_{h=1}^{H}\left(\frac{\bar{x}_{U_{h}}}{\bar{x}_{h}}\right)^{2}N_{h}^{2}(1-\frac{n_{h}}{N_{h}})\frac{1}{n_{h}}s_{uh}^{2}
\end{equation}

This is similar to a sum of $\hat{V}(\hat{t}_{yr})$ in section 3.1

\vspace{5}

\subsection{5.6.3 Discussion}
\begin{itemize}
  \item Unless the ratio $B_{h} = \frac{t_{yh}}{t_{xh}}$ is constant across stratums, the separate ratio estimator is preferred. This is provided that the sample sizes $n_{1}, ... , n_{H}$ are all reasonably large to ensure that $\hat{t}_{yrs}$ does not suffer from a large cumulative bias.
  \item The computation of $\hat{t}_{yrc}$ requires only the knowledge of the population total $t_{x}$ whereas the computation of $\hat{t}_{yrs}$ requires the stratum totals
  \item To estimate the population mean, we simply divide $\hat{t}_{yrc}$ and $\hat{t}_{yrs}$ by population size and modify variance formulae accordingly
  \item Separate regression estimates can be constructed similarly. Combined regression estimate not tested.
\end{itemize}

\rule{1\linewidth}{0.25pt}
\end{document}
