\documentclass[10pt,landscape]{article}
\usepackage{multicol}
\usepackage{calc}
\usepackage{ifthen}
\usepackage[landscape]{geometry}
\usepackage{amsmath,amsthm,amsfonts,amssymb}
\usepackage{color,graphicx,overpic}
\usepackage{hyperref}


\pdfinfo{
  /Title (example.pdf)
  /Creator (TeX)
  /Producer (pdfTeX 1.40.0)
  /Author (Khoo Wu Zhe Samuel)
  /Subject (Example)
  /Keywords (pdflatex, latex,pdftex,tex)}

% This sets page margins to .5 inch if using letter paper, and to 1cm
% if using A4 paper. (This probably isn't strictly necessary.)
% If using another size paper, use default 1cm margins.
\ifthenelse{\lengthtest { \paperwidth = 11in}}
    { \geometry{top=.5in,left=.5in,right=.5in,bottom=.5in} }
    {\ifthenelse{ \lengthtest{ \paperwidth = 297mm}}
        {\geometry{top=1cm,left=1cm,right=1cm,bottom=1cm} }
        {\geometry{top=1cm,left=1cm,right=1cm,bottom=1cm} }
    }

% Turn off header and footer
\pagestyle{empty}

% Redefine section commands to use less space
\makeatletter
\renewcommand{\section}{\@startsection{section}{1}{0mm}%
                                {-1ex plus -.5ex minus -.2ex}%
                                {0.5ex plus .2ex}%x
                                {\normalfont\large\bfseries}}
\renewcommand{\subsection}{\@startsection{subsection}{2}{0mm}%
                                {-1explus -.5ex minus -.2ex}%
                                {0.5ex plus .2ex}%
                                {\normalfont\normalsize\bfseries}}
\renewcommand{\subsubsection}{\@startsection{subsubsection}{3}{0mm}%
                                {-1ex plus -.5ex minus -.2ex}%
                                {1ex plus .2ex}%
                                {\normalfont\small\bfseries}}
\makeatother

% Define BibTeX command
\def\BibTeX{{\rm B\kern-.05em{\sc i\kern-.025em b}\kern-.08em
    T\kern-.1667em\lower.7ex\hbox{E}\kern-.125emX}}

% Don't print section numbers
\setcounter{secnumdepth}{0}


\setlength{\parindent}{0pt}
\setlength{\parskip}{0pt plus 0.5ex}

%My Environments
\newtheorem{example}[section]{Example}
% -----------------------------------------------------------------------

\begin{document}
\raggedright
\footnotesize
\begin{multicols}{3}


% multicol parameters
% These lengths are set only within the two main columns
%\setlength{\columnseprule}{0.25pt}
\setlength{\premulticols}{1pt}
\setlength{\postmulticols}{1pt}
\setlength{\multicolsep}{1pt}
\setlength{\columnsep}{2pt}

\begin{center}
     \Large{\underline{Part 6 - Single-Stage Cluster Sampling}} \\
\end{center}

\section{\underline{6.1 Description}}

Cluster sampling refers to the selection of a sample of clusters from the totality of all clusters. If all the elements in the selected clusters are surveyed, we call the method single-stage cluster sampling.

\textbf{Reasons for using cluster sampling:}
\begin{enumerate}
  \item Direct sampling of the elements may not be possible or feasible. For example, the lack of a listing of all population elements
  \item Cluster sampling is usually more convenient
  \item Cost considerations
\end{enumerate}

\textbf{Strata vs Cluster}
\begin{enumerate}
  \item In stratified sampling, we sample from each stratum. In cluster sampling, we select a sample of clusters.
  \item An ideal stratum is internally homogenous while being heterogeneous between strata. An ideal cluster has most of the variation within itself, and little variation between clusters.
\end{enumerate}

\section{\underline{6.2 Notation}}
\begin{itemize}
  \item[$N$] number of clusters in the population
  \item[$M_{i}$] number of elements in cluster $i$ (size of cluster $i$)
  \item[$M$] = $\sum_{i=1}^{N}M_{i}$
  \item total number of elements in the population (population size)
  \item[$t_{i}$] = $\sum_{j=1}^{M_{i}}y_{ij}$
  \item y-total in cluster $i$
  \item[$t$] = $\sum_{i=1}^{N}t_{i} = \sum_{i=1}^{N}\sum_{j=1}^{M_{i}}y_{ij}$
  \item population total
  \item the variance of variable $y$ in stratum $h$
  \item[$\bar{y}_{U}$] = $\frac{t}{M}$
  \item population mean of variable $y$
  \item[$S_{t}^{2}$] = $\frac{1}{N-1}\sum_{i=1}^{N}(t_{i}-\frac{t}{N})^{2}$
  \item population variance of the cluster totals
\end{itemize}

\textbf{Sample quantities}
\begin{itemize}
  \item[$n$] number of clusters sampled
  \item[\textbf{S}] simple random sample of $n$ clusters
  \item[$\bar{t}$] = $\frac{1}{n}\sum_{i \in \textbf{S}}t_{i}$
  \item sample mean of the cluster total $t_{i}$
  \item[$\hat{R}_{mpe}$] = $\frac{\sum_{i \in \textbf{S}}t_{i}}{\sum_{i \in \textbf{S}}M_{i}} = \frac{\sum_{i \in \textbf{S}}\sum_{j=1}^{M_{i}}y_{ij}}{\sum_{i \in \textbf{S}}M_{i}}$
  \item sample mean of variable $y$ per element
\end{itemize}

\section{\underline{6.3 Estimation of population total}}
\section{\underline{and mean}}
\subsection{6.3.1 Naive number-raised estimator}
If SRS is used to select the clusters,
\begin{center}
  $\bar{t} = \frac{1}{n}\sum_{i \in \textbf{S}}t_{i}$
\end{center}
is an unbiased estimator of the mean cluster total $\frac{1}{N}\sum_{i=1}^{N}t_{i}$
\begin{center}
  $\hat{t}_{unb} = N\bar{t}$
\end{center}
is an unbiased estimator of the population total.

\vspace{5}

\textbf{Variance of estimator:}
\begin{center}
  $V(\hat{t}_{unb}) = V(N\bar{t}) = N^{2}(1-\frac{n}{N})\frac{S_{t}^{2}}{n}$
\end{center}

\textbf{Estimated variance of estimator:}
\begin{center}
  $\hat{V}(\hat{t}_{unb}) = N^{2}(1-\frac{n}{N})\frac{s_{t}^{2}}{n}$
\end{center}
where
\begin{center}
  $s_{t}^{2} = \frac{1}{n-1}\sum_{i \in \textbf{S}}(t_{i} - \bar{t})^{2}$
\end{center}
If $M$ is known, we can estimate $\bar{y}_{U}$ by
\begin{center}
  $\hat{\bar{y}}_{unb} = \frac{\hat{t}_{unb}}{M}$
\end{center}
with $SE(\hat{\bar{y}}_{unb}) = \frac{SE(\hat{t}_{unb})}{M}$

\subsection{6.3.2 Ratio estimator}
Ratio estimator of $\bar{y}_{U}$ is $\hat{R}_{mpe}$
\begin{center}
  $\hat{V}(\hat{R}_{mpe}) = \frac{1}{\bar{M}^{2}}(1-\frac{n}{N})\frac{1}{n}\frac{1}{n-1}\sum_{i \in \textbf{S}}(t_{i} - \hat{R}_{mpe}M_{i})^{2}$
\end{center}
where $\bar{M} = \frac{1}{n}\sum_{i \in \textbf{S}}M_{i}$ is the average cluster size of the selected clusters.

\vspace{5}

If M is known, the ratio estimate of the total is:
\begin{equation}
  \begin{split}
    \hat{t}_{r} &= M \\
    &= M\hat{R}_{mpe} \\
    &= M\frac{\sum_{i \in \textbf{S}}t_{i}}{\sum_{i \in \textbf{S}}M_{i}}
  \end{split}
\end{equation}

with estimated variance
\begin{center}
  $\hat{V}(\hat{t}_{r}) = N^{2}\left(\frac{\bar{M}_{U}}{\bar{M}}\right)^{2}(1-\frac{n}{N})\frac{1}{n}\frac{1}{n-1}\sum_{i \in \textbf{S}}(t_{i} - \hat{R}_{mpe}M_{i})^{2}$
\end{center}

where $\bar{M}_{U} = \frac{1}{N}\sum_{i =1}^{N}M_{i} = \frac{M}{N}$

\subsection{Note:}
The ratio estimator differs from the native estimater when the cluster sizes are unequal. By rewriting the ratio estimator,
\begin{center}
  $\hat{t}_{r} = \frac{M/N}{\bar{M}}\hat{t}_{unb}$
\end{center}
we see that the ratio estimator adjusts the naive estimate upward/downward depending on whether the mean size of the selected clusters ($\bar{M}$) is below/above the mean size of all clusters ($\bar{M}_{U}$).

\vspace{5}

Since the cluster total ($t_{i}$) is expected to be positively correlated with the cluster size ($M_{i}$), we expect the ratio estimator to be more efficient that the naive estimator.

\rule{1\linewidth}{0.25pt}
\end{document}
